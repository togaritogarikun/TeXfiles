%%%%%%%%%%%%%%%%%%%%%%%%%%%%%%%%%%%%%%%%%%%%%%%%%%%%%%%%%%%%
%                                                          %
%    slide.tex                                             %
%    ~~~~~~~~~~~~~~~~~~~~~~~~~                             %
%    This file is the template for slide-like material.    %
%                                                          %
%    made by Yudai YAMAKAWA                                %
%    May 4, 2020                                           %
%    last update: May 5, 2020                              %
%                                                          %
%    GitHub: https://github.com/togaritogarikun            %
%                                                          %
%%%%%%%%%%%%%%%%%%%%%%%%%%%%%%%%%%%%%%%%%%%%%%%%%%%%%%%%%%%%
%\documentclass[11pt,twocolumn,dvipdfmx,platex,slide,papersize]{jsarticle}
%\documentclass[disablejfam,landscape]{jsarticle}
%\documentclass[18pt,landscape]{jsarticle}
%\documentclass[43pt,a4paper]{article}
\documentclass[10pt,dvipdfmx,landscape]{jsarticle}

%%% package %%%%%%%%%%%%%%%%%%%%%%%%%%%%%%%%%%%%%%%%%%%%%%%%
\usepackage{mytexsty_v6}
\lhead{\rightmark}

%%%%%%%%%%%%%%%%%%%%%%%%%%%%%%%%%%%%%%%%%%%%%%%%%%%%%%%%%%%%
\begin{document}
%%%%%%%%%%%%%%%%%%%%%%%%%%%%%%%%%%%%%%%%%%%%%%%%%%%%%%%%%%%%
%%% Header
%
%
%
%\twocolumn[
\noindent
{\footnotesize
Ohnishi lab., plasma study group    %%% [your lab, your study group]
}

\noindent
%\begin{center}
%%%% Title %%%%%%%%%%%%%%%%%%%%%%%%%%%%%%%%%%%%%%%%%%%%%%%%%
	\textbf{\Large
		Introduction to Plasma Physics, Goldston    %%% [book name you use]
	}

	\vspace{1ex}

\noindent
%%%% Jornal %%%%%%%%%%%%%%%%%%%%%%%%%%%%%%%%%%%%%%%%%%%%%%%%
	\textrm{\small 
		Yudai YAMAKAWA (May 19, 2020)    %%% [your name, date when you presentate]
	}


%\end{center}
%\normalsize

\vspace{4ex}
%]

\begin{boxnote}
	今日の内容
	\vspace{1mm}
	\begin{itemize}
		\item デバイ遮蔽
		\item プローブのプラズマへの挿入
		\item 一様磁場中での荷電粒子の運動
	\end{itemize}
\end{boxnote}

%\footnotesize
%%%%%%%%%%%%%%%%%%%%%%%%%%%%%%%%%%%%%%%%%%%%%%%%%%%%%%%%%%%%
\newpage
\setcounter{section}{1}
\setcounter{subsection}{6}
\subsection{DEBYE SHIELDING}
\begin{screen}
	まとめの一文 / summary
\end{screen}
%
aaaa  
%%%%%%%%%%%%%%%%%%%%%%%%%%%%%%%%%%%%%%%%%%%%%%%%%%%%%%%%%%%%
\newpage
\begin{screen}
	「熱平衡状態」であるとは?
\end{screen}
%
\begin{itemize}
	\item 「熱平衡%\footnote{熱平衡プラズマ (熱プラズマ) や非平衡プラズマの「平衡」は必ずしも気体分子運動論が言うところの「熱平衡」とは一致しない. 「熱平衡プラズマ」は, 電子, イオン, そして中性粒子が平衡状態にある事を意味する. 非平衡プラズマでは, 電子及びイオンそれぞれが熱平衡状態となっている. 熱平衡プラズマは, 大気圧かそれ以上の圧力下で発生しやすい. これは, 平均自由工程が小さく, 衝突が頻繁に起こる様になる為である. }
	状態」であるとは, ある系\footnote{たくさんの粒子が存在している特定領域}の中から取り出したどの部分も, 他の部分と同じ状態で区別出来ない様な状態と言える
	\item 熱接触している 2 つの物体が熱平衡状態にあるとき, 共有している釣り合いを表す物理量が「温度」である
	\item つまり温度とは, 系を特徴づける為に各粒子が持つ特徴を平均化したものである
	\item 温度を厳密に決める事が出来るのは, 気体 (プラズマ) が熱平衡状態にあるときのみである
\end{itemize}
%%%%%%%%%%%%%%%%%%%%%%%%%%%%%%%%%%%%%%%%%%%%%%%%%%%%%%%%%%%%
\newpage
\begin{screen}
	ボルツマン因子
\end{screen}
%
\begin{itemize}
	\item ボルツマン因子は, 
熱平衡状態にある形に於いて, 
あるエネルギー状態 $W_r$ が発現する相対的な確率を定める因子である
	\item カノニカルアンサンブル\footnote{外とエネルギーを自由にやり取り出来る「閉鎖系」の集まり}で記述される系に対して用いられる
	\item いま, プラズマの温度は時空間変化しない等温状態であると仮定し, 分布関数が熱浴\footnote{エネルギーのやり取りを行う事の出来る大きな自由度を持つ系のこと. 他の系と接触してもその状態を乱される事がなく, 常に熱平衡状態に保たれている系. 熱源とも呼ばれる.}であると考える. 又, エネルギーとして運動エネルギーとポテンシャルエネルギーを考える
	\item この場合, ボルツマン因子は次の式で表される:
	\begin{equation}
		\exp \left( - \frac{m v^{2} / 2 + q \phi}{T} \right)
	\end{equation}
\end{itemize} 
%%%%%%%%%%%%%%%%%%%%%%%%%%%%%%%%%%%%%%%%%%%%%%%%%%%%%%%%%%%%
\newpage
\begin{screen}
	速度分布関数から密度の導出
\end{screen}
%
\begin{itemize}
	\item 速度分布関数は, 3 次元速度ベクトル空間中の検査体積 $\dd ^{3} = \dd v_x \dd v_y \dd v_z$ 中に入る分子数である
	\item 従って, 速度分布関数を速度空間の全範囲に渡って積分 (速度空間にある微小体積の中の分子数の総和) する事によって, 全分子数が求まる
\end{itemize}
%%%%%%%%%%%%%%%%%%%%%%%%%%%%%%%%%%%%%%%%%%%%%%%%%%%%%%%%%%%%
\newpage
\begin{screen}
	$n \propto \exp ( - q \phi / T )$ の持つ意味
\end{screen}
%
aaaa
%%%%%%%%%%%%%%%%%%%%%%%%%%%%%%%%%%%%%%%%%%%%%%%%%%%%%%%%%%%%
\newpage
\begin{screen}
	デバイ遮蔽
\end{screen}
%
aaaa
%%%%%%%%%%%%%%%%%%%%%%%%%%%%%%%%%%%%%%%%%%%%%%%%%%%%%%%%%%%%
\newpage
\begin{screen}
	$T_{\rm e} \neq T_{\rm i}$
\end{screen}
%
aaaa
%%%%%%%%%%%%%%%%%%%%%%%%%%%%%%%%%%%%%%%%%%%%%%%%%%%%%%%%%%%%
\newpage
\begin{screen}
	\vspace{1mm}
	Poisson 方程式の解
\end{screen}
%
aaaa 
%%%%%%%%%%%%%%%%%%%%%%%%%%%%%%%%%%%%%%%%%%%%%%%%%%%%%%%%%%%%
\newpage
\begin{screen}
	デバイ長の導出 (3)\\
	\vspace{1mm}
	デバイ長
\end{screen}
%
aaaa 
%%%%%%%%%%%%%%%%%%%%%%%%%%%%%%%%%%%%%%%%%%%%%%%%%%%%%%%%%%%%
\newpage
\begin{screen}
	
\end{screen}
%
aaaa 
%%%%%%%%%%%%%%%%%%%%%%%%%%%%%%%%%%%%%%%%%%%%%%%%%%%%%%%%%%%%
\newpage
\begin{screen}
	
\end{screen}
%
aaaa   
%%%%%%%%%%%%%%%%%%%%%%%%%%%%%%%%%%%%%%%%%%%%%%%%%%%%%%%%%%%%
\newpage
\subsection{MATERIAL PROBES IN A PLASMA}
\begin{screen}
	プラズマと壁との間に形成される層: シース
\end{screen}
%
aaaa\footnote{bbbb} 
%%%%%%%%%%%%%%%%%%%%%%%%%%%%%%%%%%%%%%%%%%%%%%%%%%%%%%%%%%%%
\newpage
\begin{screen}
	浮遊電位
\end{screen}
%
aaaa  
%%%%%%%%%%%%%%%%%%%%%%%%%%%%%%%%%%%%%%%%%%%%%%%%%%%%%%%%%%%%
\newpage
\begin{screen}
	イオン飽和電流
\end{screen}
%
aaaa  
%%%%%%%%%%%%%%%%%%%%%%%%%%%%%%%%%%%%%%%%%%%%%%%%%%%%%%%%%%%%
\newpage
\begin{screen}
	プローブ診断法
\end{screen}
%
aaaa  
%%%%%%%%%%%%%%%%%%%%%%%%%%%%%%%%%%%%%%%%%%%%%%%%%%%%%%%%%%%%
\newpage
\setcounter{section}{2}
\subsection{GYRO-MOTION}
\begin{screen}
	プラズマが持つ共通の性質
\end{screen}
%
aaaa
%%%%%%%%%%%%%%%%%%%%%%%%%%%%%%%%%%%%%%%%%%%%%%%%%%%%%%%%%%%%
\newpage
\begin{screen}
	一様磁場中の荷電粒子に対する運動方程式
\end{screen}
%
aaaa  
%%%%%%%%%%%%%%%%%%%%%%%%%%%%%%%%%%%%%%%%%%%%%%%%%%%%%%%%%%%%
\newpage
\begin{screen}
	サイクロトロン周波数
\end{screen}
%
aaaa  
%%%%%%%%%%%%%%%%%%%%%%%%%%%%%%%%%%%%%%%%%%%%%%%%%%%%%%%%%%%%
\newpage
\begin{screen}
	一様磁場中の荷電粒子の位置
\end{screen}
%
aaaa  
%%%%%%%%%%%%%%%%%%%%%%%%%%%%%%%%%%%%%%%%%%%%%%%%%%%%%%%%%%%%
\newpage
\begin{screen}
	ラーマー半径
\end{screen}
%
aaaa  
%%%%%%%%%%%%%%%%%%%%%%%%%%%%%%%%%%%%%%%%%%%%%%%%%%%%%%%%%%%%
\newpage
\begin{screen}
	旋回中心の運動
\end{screen}
%
aaaa
%%%%%%%%%%%%%%%%%%%%%%%%%%%%%%%%%%%%%%%%%%%%%%%%%%%%%%%%%%%%
\newpage
\begin{screen}
	「反磁性」
\end{screen}
%
aaaa  
%%%%%%%%%%%%%%%%%%%%%%%%%%%%%%%%%%%%%%%%%%%%%%%%%%%%%%%%%%%%
\newpage
\begin{screen}
	磁化プラズマが持つ空間・時間スケール
\end{screen}
%
aaaa  
%%%%%%%%%%%%%%%%%%%%%%%%%%%%%%%%%%%%%%%%%%%%%%%%%%%%%%%%%%%%
\newpage
\begin{screen}
	
\end{screen}
%
aaaa    
%%%%%%%%%%%%%%%%%%%%%%%%%%%%%%%%%%%%%%%%%%%%%%%%%%%%%%%%%%%%
\end{document}
%%%%%%%%%%%%%%%%%%%%%%%%%%%%%%%%%%%%%%%%%%%%%%%%%%%%%%%%%%%%
